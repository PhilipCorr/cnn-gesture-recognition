\documentclass[]{weekly-report}
\usepackage{hyperref}
 
%%%%%%%%%%%%%%%%%%%%%%%%%%%%%%%%%%%
%%%%%%%%%%%%%% BEGINNING %%%%%%%%%%%%%%
%%%%%%%%%%%%%%%%%%%%%%%%%%%%%%%%%%%

\begin{document}


%%%%%%%%%%%%%%%%%%%%%%
%%% Input your name, student number, 
%%% project and report details

\def\studentname{Philip Corr}
\def\projecttitle{ConvNets for iOS Gesture Recognition Applications}
\def\ucdstudentnumber{12318581}
\def\weeklyreportnumber{9}
\maketitle

%%%%%%%%%%%%%%%%%%%%%%ss
%%% First Section  

\section{Update}

\begin{itemize}

\item Have Permission to record data

\item All touch points in same file in DB browser. How to extract which point is belong to which subject?

\item Would like to be further ahead with the tensorflow stuff. Will make a big push to cover ground before presentation. What should I do? Still go for iOS device running MNIST? Essintially implementing another app to do this? Will have to check out tensorflow and iOS compatibility more.
	
\end{itemize}

\section{Items for discussion}

\begin{itemize}
\item See update...
 

\end{itemize}

\section{Plan for Next Week}



\section{Meeting Notes}




%%%%%%%%%%%%%%%%%%%%%%
%%% Bibliography

\bibliography{report-biblio}{}
\bibliographystyle{IEEEtran}

%%%%%%%%%%%%%%%%%%%%%%%%%%%%%%%%%%%
%%%%%%%%%%%%%% END %%%%%%%%%%%%%%%%%%
%%%%%%%%%%%%%%%%%%%%%%%%%%%%%%%%%%%

\label{last_page}

 \end{document} 